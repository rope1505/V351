\begin{table}[!htp]
\centering
\caption{Die gemessenen Amplituden der drei Schwingungen, sowie deren Normierung auf die erste Oberwelle.}
\label{tab:amplituden}
\begin{tabular}{c c c c c c c c }
\toprule
 & \multicolumn{2}{c}{Rechteck} & \multicolumn{2}{c}{Dreieck}  &  \multicolumn{3}{c}{Sägezahn}  \\
 \cmidrule(lr){2-3} 
  \cmidrule(lr){4-5}
   \cmidrule(lr){6-8}
{$n$} & {$U_n$ / mV} & {$\frac{U_n}{U_1}$} & {$U_n$ / mV} & {$\frac{U_n}{U_1}$} & {$n$} & {$U_n$ / mV} & {$\frac{U_n}{U_1}$}  \\
\midrule
  1 & 2000 & 1.000 & 2800  & 1.000 & 1 &  2160 & 1.000  \\
  3 &712 & 0.356   & 272 & 0.097   &2 &  1030 & 0.477\\
   5 &432 & 0.216  & 102 & 0.036   &3 &  656 & 0.304 \\
   7 &288 & 0.144  & 50 & 0.018    &4 &  552 & 0.256 \\
   9 &216 & 0.108  & 28 & 0.010    &5 &  448 & 0.207 \\
   11&200 & 0.100  & 22 & 0.008    &6 &  384 & 0.178 \\
   13& 168 & 0.084 &    &          &7 &  312 & 0.144 \\
   15& 144 & 0.072 &    &          &8 &  264 & 0.122 \\
   17& 112 & 0.056 &    &          &9 &  248 & 0.115 \\
   19& 104 & 0.052 &    &          &10 & 232 & 0.107 \\
\bottomrule
\end{tabular}
\end{table}