\subsection{Vorbereitung - Bestimmung der Fourierkoeffizienten}
\label{sec:vorbereitung}
Zur Vorbereitung auf das Experiment sind die Fourierkoeffizienten von drei Schwingungsformen, einer Reckteck-, einer Sägezahn-
und einer Dreieck, zu bestimmen. Dabei ist es von Vorteil die Schwingungen als gerade oder ungerade Funktionen zu definieren, so 
dass Koeffizienten wegfallen. Die Berechung dieser erfolgt jeweils über GLEICHUNGAK und GLEICHUNGBK, woraus sich dann nach Gleichung
GLEICHUNGFOURIER jeweils die zugehörige Fourierreihe berechnet.
    \subsection{Rechteckschwinung}
    Die Rechteckschwingung wird als
    \begin{equation}
    \label{eqn:rechteck}
    g(t) = \begin{cases}
                1 & -\frac{T}{2} < t < 0 \\
                0 & 0 < t < \frac{T}{2} 
            \end{cases}
    \end{equation}        
    definiert. Die Fourierkoeffizienten dieser Funktion sind
    \\
    \centerline{$a_n = 0$} 
    \\
    und
    \\ 
    \centerline{$b_n = \begin{cases}
                            0 & n \text{ gerade} \\
                            \frac{4}{\pi n } & n \text{ ungerade}
                        \end{cases}$.}
                        \\
    Die zugehörige Fourierreihe ist daher
    \begin{equation}
    \label{eqn:fourierrechteck}
    f(t) = \sum_{n=1}^{\infty}  \frac{4}{\pi n } \sin(n \frac{2 \pi}{T} t),
    \end{equation}
    wobei $n$ nur ungerade Werte annimmt.

    \subsection{Sägezahnschwingung}
    Die Funktion, die die Sägezahnschwingung beschreibt lautet:
    \begin{equation}
    \label{eqn:saegezahn}
    g(t) = \begin{cases}
            \lvert t \rvert & -T \leq t < T \\
            0 & t = T 
            \end{cases}.
    \end{equation}
    Die Fourierkoeffizienten dieser Funktion sind
    \\ 
    \centerline{$a_n = \frac{2T}{\pi n} \cos(\pi n)$}
    \\
    und 
    \\ 
    \centerline{$b_n = 0$.}
    Daher ist die zugehörige Fourierreihe
    \begin{equation}
    \label{eqn:fouriersäge} 
    f(t) = \sum_{i=1}^{\infty} \frac{2 T}{\pi n} \cos(\frac{ 2 \pi n}{T}t).
    \end{equation}

    \subsection{Dreieckschwingung}
    Die Funtktion zur Beschreibung der Dreieckschwingung ist definiert als
    \begin{equation}
    \label{eqn:dreieck}
    g(t) = \begin{cases} 
            - \frac {2 A} {T} & - \frac {T}{2} < t < 0 \\
            \frac {2 A} {T} & 0 < t < \frac {T}{2}
            \end{cases},
    \end{equation}
    wobei $A$ die Amplitude der Schwingung ist.
    Die Fourierkoeffizienten zu dieser Funktion sind
    \\
    \centerline{$a_n = \begin{cases}
                        0 & n \text{ gerade} \\
                        - \frac{8 A}{(\pi n)^2} & n \text{ungerade} \end{cases}$}
                        \\
    und   
    \\         
    \centerline{$b_n = 0$.} 
    \\      
    Die dazugehörige Fourierreihe ist daher
    \begin{equation}
    \label{eqn:fourierdrei}
    f(t) = \sum_{i=1}^{\infty} - \frac{8 A}{(\pi n)^2} \cos(\frac{2 \pi n}{T}t),    
    \end{equation}
    wobei $n$ nur ungerade Werte annimmt.
 

