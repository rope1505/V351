\section{Diskussion}
\label{sec:Diskussion}
Bei der Fourieranalyse sind die Steigungen der drei Ausgleichsgeraden näher zu betrachten, insbesondere ihr Vergleich zu den 
Theoriewerten. Bei der Rechteckschwingung, sowie bei der Sägezahnschwingung soll eine Steigung der Ausgleichsgeraden von $-1$;
bei der Dreieckschwingung eine Steigung von $-2$. Die Steigungen der Geraden mit den relativen Abweichungen von den Theoriewerten 
sind: 
\\ \\
\centerline{$a_\text{Rechteck} = ( -1.251 \pm 0.036 ) $}
\centerline{Abweichung: $20.06\%$}
\\ \\
\centerline{$ a_\text{Sägezahn} = (- 0.966 \pm 0.021)$}
\centerline{Abweichung: $-3.52 \% $}
\\ \\
\centerline{$a_\text{Dreieck} = ( -2.709 \pm 0.144 )$}
\centerline{Abweichung: $26.17 \%$}
\\ \\
