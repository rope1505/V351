\section{Diskussion}
\label{sec:Diskussion}
Bei der Fourieranalyse sind die Steigungen der drei Ausgleichsgeraden näher zu betrachten, insbesondere ihr Vergleich zu den 
Theoriewerten. Bei der Rechteckschwingung, sowie bei der Sägezahnschwingung soll eine Steigung der Ausgleichsgeraden von $-1$;
bei der Dreieckschwingung eine Steigung von $-2$. Die Steigungen der Geraden mit den relativen Abweichungen von den Theoriewerten 
sind: 
\\ \\
\centerline{$a_\text{Rechteck} = ( -1.002 \pm 0.021 ) $}
\\ \\
mit einer Abweichung von $0.2\%$,
\\ \\
\centerline{$ a_\text{Sägezahn} = (- 0.966 \pm 0.021)$}
\\ \\
mit einer Abweichung von $-3.5 \% $ und 
\\ \\
\centerline{$a_\text{Dreieck} = (-2.041 \pm 0.037 )$}
\\ \\
mit einer Abweichung von $2.1 \%$ vom  jeweiligen Theoriewert.
Die relativen Abweichungen der Steigungen sind alle sehr gering, was auf eine allgemein sehr genaue Messung schließen lässt.
Besonders die sehr geringe Abweichung bei der Reckteckschwingung sticht hierbei hervor. Da allerdings die Messung bei allen 
Schwingungen gleich durchgeführt wurde, ist davon auszugehen, dass es sich bei diesem sehr gutem Ergebnis lediglich um eine 
zufällig bessere Messung handelt. Bei der Analyse der Dreieckschwingung konnten zudem nur sechs, statt wie bei den anderen 
Messungen zehn, Messwerte aufgenommen werden. Dies liegt vor allem daran, dass die Koeffizienten der Dreieckschwingung mit 
$\frac{1}{n^2}$ abfallen und daher wesentlich schneller sehr klein werden, als bei den anderen beiden Schwingungen, deren 
Koeffizienten bloß mit $\frac{1}{n}$ abfallen. Somit waren weitere Messwerte mit den verwendeten Messgeräten nicht mehr 
zuverlässig messbar; zur Messung weiterer Werte werden somit genauere Messgeräte erforderlich.

Bei der Fouerier-Synthese traten verschieden Ungenauigkeiten im Zusammenhang mit dem verwendeten Oberwellengenerator auf. 
Bei diesem lag ein interner Wackelkontakt vor, was die genaue Justierung der Amplituden sehr erschwerte. Ganz genaues
Einstellen war oftmals aufgrund dessen nicht möglich; die Abweichungen liegen aber im Rahmen. Lediglich bei der Dreieckschwingung
traten, da hier früh sehr kleine Amplituden eingestellt werden müssen, größere Schwankungen auf, die aber auch nicht eindeutig 
bestimmbar sind. Dennoch fallen die synthesierten Graphen wie zu erwarten aus. Das sogennante Gibb'sche Phänomen ist bei der 
Rechteckschwingung, sowie bei der Sägezahnschwingung deutlich an den verschieden Nebenamplituden zu erkennen. Dies tritt auf, da
hier im Experiment mit einer endlichen Anzahl an Oberwellen gearbeitet wird. Erst bei unendlich vielen, einwirkenden Oberwellen
verschwindet dieses Phänomen. Da dies im Experiment allerdings nicht umsetzbar ist, sind diese Phänomene zu erwarten.
Bei der Rechteckschwingung sind diese Nebenamplituden sehr stark ausgeprägt, lassen aber dennoch auf die Form der Rechteckschwingung schließen.
Die Unstetigkeitsstellen der Sägezahnschwingung sind nur durch eine sehr steile rechte Flanke dargestellt, was 
das Gesamtbild der Synthese der Sägezahnschwingung verfälscht. Dennoch lässt sich sagen, dass die Synthese der Rechteck- und Sägezahnschwingung
gut verlaufen ist.
Da bei der Synthese der Dreieckschwingung nur zwei Oberwellen Verwendung fanden, ist das Ergebnis entsprechend ungenau.
Die Nebenamplitude an den linken Flanken dieser führt zu einer starken Verfälschung dieser. Des Weiteren sind die 
Knickstellen an den Amplituden nicht so spitz, wie es bei einer Dreieckschwingung zu erwarten wäre. Die Synthese der 
Dreieckschwingung kann durch das Hinzufügen weiterer Oberwellen deutlich verbessert werden, wozu aber bessere Geräte von Nöten sind.


