\section{Durchführung}
\label{sec:Durchführung}
\subsection{Vorbereitung - Bestimmung der Fourierkoeffizienten}
\label{sec:vorbereitung}

\subsection{Fourieranalyse}

In diesem Versuchsteil wird ein Oszilloskop an einen Funktionsgenerator angeschlossen.
Am Funktionsgenerator werden nacheinander eine Rechteck-, Dreieck- und Sägezahnschwingung eingstellt.
Das Oszilloskop führt eine Fourieranalyse durch und zeigt ein Frequenzspektrum.
Mit dem Cursor werden die Amplituden abgelesen und anschließend notiert.

\subsection{Fouriersynthese}

Zunächst werden zwei Ausgänge eines Oberwellengenerator an das Oszilloskop angeschlossen, welches in den XY-Betrieb umgeschaltet wird.
Die so entstehenden Lissajous-Figuren werden genutzt, um die beiden Signale phasengleich zu schlaten.
Diese Prozedur wird mit allen Ausgängen durchgeführt.
Dabei werden die Amplituden maximal eingestellt, um genaure Einstellungen vornehmen zu können.
%sägezahn, rechteck > 1/n
%dreieck > 1/n²
Im Weiteren werden die Amplituden so eingestellt, dass sie dem Proportionalitätsfaktoren der zu synthesierenden Schwingungen passen. Dieser wird zuvor in der Vorbereitung berechnet.
Für diese Konfiguration wird ein Voltmeter genutzt.

Das Oszilloskop wird nun erneut in den XT-Betrieb umgeschaltet und die einzelnen Oberwellen angeschlossen.
Die Phase wird mithilfe von zwei Schaltern um 90° beziehungsweise 180° verschoben, bis die angezeigte Schwingungen die jeweils Gewünschte gut approximiert.

