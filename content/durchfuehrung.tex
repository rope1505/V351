\section{Durchführung}
\label{sec:Durchführung}
\subsection{Vorbereitung - Bestimmung der Fourierkoeffizienten}
\label{sec:vorbereitung}
Zur Vorbereitung auf das Experiment sind die Fourierkoeffizienten von drei Schwingungsformen, einer Reckteck-, einer Sägezahn-
und einer Dreieck, zu bestimmen. Dabei ist es von Vorteil die Schwingungen als gerade oder ungerade Funktionen zu definieren, so 
dass Koeffizienten wegfallen. Die Berechung dieser erfolgt jeweils über GLEICHUNGAK und GLEICHUNGBK, woraus sich dann nach Gleichung
GLEICHUNGFOURIER jeweils die zugehörige Fourierreihe berechnet.
    \subsection{Rechteckschwinung}
    Die Rechteckschwingung wird als
    \begin{equation}
    \label{eqn:rechteck}
    g(t) = \begin{cases}
                1 & -\frac{T}{2} < t < 0 \\
                0 & 0 < t < \frac{T}{2} 
            \end{cases}
    \end{equation}        
    definiert. Die Fourierkoeffizienten dieser Funktion sind
    \\
    \centerline{$a_n = 0$} 
    \\
    und
    \\ 
    \centerline{$b_n = \begin{cases}
                            0 & n \text{ gerade} \\
                            \frac{4}{\pi n } & n \text{ ungerade}
                        \end{cases}$.}
                        \\
    Die zugehörige Fourierreihe ist daher
    \begin{equation}
    \label{eqn:fourierrechteck}
    f(t) = \sum_{n=1}^{\infty}  \frac{4}{\pi n } \sin(n \frac{2 \pi}{T} t),
    \end{equation}
    wobei $n$ nur ungerade Werte annimmt.

    \subsection{Sägezahnschwingung}
    Die Funktion, die die Sägezahnschwingung beschreibt lautet:
    \begin{equation}
    \label{eqn:saegezahn}
    g(t) = \begin{cases}
            \lvert t \rvert & -T \leq t < T \\
            0 & t = T 
            \end{cases}.
    \end{equation}
    Die Fourierkoeffizienten dieser Funktion sind
    \\ 
    \centerline{$a_n = \frac{2T}{\pi n} \cos(\pi n)$}
    \\
    und 
    \\ 
    \centerline{$b_n = 0$.}
    Daher ist die zugehörige Fourierreihe
    \begin{equation}
    \label{eqn:fouriersäge} 
    f(t) = \sum_{i=1}^{\infty} \frac{2 T}{\pi n} \cos(\frac{ 2 \pi n}{T}t).
    \end{equation}

    \subsection{Dreieckschwingung}
    Die Funtktion zur Beschreibung der Dreieckschwingung ist definiert als
    \begin{equation}
    \label{eqn:dreieck}
    g(t) = \begin{cases} 
            - \frac {2 A} {T} & - \frac {T}{2} < t < 0 \\
            \frac {2 A} {T} & 0 < t < \frac {T}{2}
            \end{cases},
    \end{equation}
    wobei $A$ die Amplitude der Schwingung ist.
    Die Fourierkoeffizienten zu dieser Funktion sind
    \\
    \centerline{$a_n = \begin{cases}
                        0 & n \text{ gerade} \\
                        - \frac{8 A}{(\pi n)^2} & n \text{ungerade} \end{cases}$}
                        \\
    und   
    \\         
    \centerline{$b_n = 0$.} 
    \\      
    Die dazugehörige Fourierreihe ist daher
    \begin{equation}
    \label{eqn:fourierdrei}
    f(t) = \sum_{i=1}^{\infty} - \frac{8 A}{(\pi n)^2} \cos(\frac{2 \pi n}{T}t),    
    \end{equation}
    wobei $n$ nur ungerade Werte annimmt.
 



\subsection{Fourieranalyse}

In diesem Versuchsteil wird ein Oszilloskop an einen Funktionsgenerator angeschlossen.
Am Funktionsgenerator werden nacheinander eine Rechteck-, Dreieck- und Sägezahnschwingung eingstellt.
Das Oszilloskop führt eine Fourieranalyse durch und zeigt ein Frequenzspektrum.
Mit dem Cursor werden die Amplituden abgelesen und anschließend notiert.

\subsection{Fouriersynthese}

Zunächst werden zwei Ausgänge eines Oberwellengenerator an das Oszilloskop angeschlossen, welches in den XY-Betrieb umgeschaltet wird.
Die so entstehenden Lissajous-Figuren werden genutzt, um die beiden Signale phasengleich zu schlaten.
Diese Prozedur wird mit allen Ausgängen durchgeführt.
Dabei werden die Amplituden maximal eingestellt, um genaure Einstellungen vornehmen zu können.
%sägezahn, rechteck > 1/n
%dreieck > 1/n²
Im Weiteren werden die Amplituden so eingestellt, dass sie dem Proportionalitätsfaktoren der zu synthesierenden Schwingungen passen. Dieser wird zuvor in der Vorbereitung berechnet.
Für diese Konfiguration wird ein Voltmeter genutzt.

Das Oszilloskop wird nun erneut in den XT-Betrieb umgeschaltet und die einzelnen Oberwellen angeschlossen.
Die Phase wird mithilfe von zwei Schaltern um 90° beziehungsweise 180° verschoben, bis die angezeigte Schwingungen die jeweils Gewünschte gut approximiert.

